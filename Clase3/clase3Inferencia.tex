% Options for packages loaded elsewhere
\PassOptionsToPackage{unicode}{hyperref}
\PassOptionsToPackage{hyphens}{url}
%
\documentclass[
  ignorenonframetext,
]{beamer}
\usepackage{pgfpages}
\setbeamertemplate{caption}[numbered]
\setbeamertemplate{caption label separator}{: }
\setbeamercolor{caption name}{fg=normal text.fg}
\beamertemplatenavigationsymbolsempty
% Prevent slide breaks in the middle of a paragraph
\widowpenalties 1 10000
\raggedbottom
\setbeamertemplate{part page}{
  \centering
  \begin{beamercolorbox}[sep=16pt,center]{part title}
    \usebeamerfont{part title}\insertpart\par
  \end{beamercolorbox}
}
\setbeamertemplate{section page}{
  \centering
  \begin{beamercolorbox}[sep=12pt,center]{part title}
    \usebeamerfont{section title}\insertsection\par
  \end{beamercolorbox}
}
\setbeamertemplate{subsection page}{
  \centering
  \begin{beamercolorbox}[sep=8pt,center]{part title}
    \usebeamerfont{subsection title}\insertsubsection\par
  \end{beamercolorbox}
}
\AtBeginPart{
  \frame{\partpage}
}
\AtBeginSection{
  \ifbibliography
  \else
    \frame{\sectionpage}
  \fi
}
\AtBeginSubsection{
  \frame{\subsectionpage}
}
\usepackage{lmodern}
\usepackage{amssymb,amsmath}
\usepackage{ifxetex,ifluatex}
\ifnum 0\ifxetex 1\fi\ifluatex 1\fi=0 % if pdftex
  \usepackage[T1]{fontenc}
  \usepackage[utf8]{inputenc}
  \usepackage{textcomp} % provide euro and other symbols
\else % if luatex or xetex
  \usepackage{unicode-math}
  \defaultfontfeatures{Scale=MatchLowercase}
  \defaultfontfeatures[\rmfamily]{Ligatures=TeX,Scale=1}
\fi
\usetheme[]{Marburg}
\usefonttheme{structurebold}
% Use upquote if available, for straight quotes in verbatim environments
\IfFileExists{upquote.sty}{\usepackage{upquote}}{}
\IfFileExists{microtype.sty}{% use microtype if available
  \usepackage[]{microtype}
  \UseMicrotypeSet[protrusion]{basicmath} % disable protrusion for tt fonts
}{}
\makeatletter
\@ifundefined{KOMAClassName}{% if non-KOMA class
  \IfFileExists{parskip.sty}{%
    \usepackage{parskip}
  }{% else
    \setlength{\parindent}{0pt}
    \setlength{\parskip}{6pt plus 2pt minus 1pt}}
}{% if KOMA class
  \KOMAoptions{parskip=half}}
\makeatother
\usepackage{xcolor}
\IfFileExists{xurl.sty}{\usepackage{xurl}}{} % add URL line breaks if available
\IfFileExists{bookmark.sty}{\usepackage{bookmark}}{\usepackage{hyperref}}
\hypersetup{
  pdftitle={Estadística Inferencial},
  pdfauthor={Jackson M'coy Romero Plasencia},
  hidelinks,
  pdfcreator={LaTeX via pandoc}}
\urlstyle{same} % disable monospaced font for URLs
\newif\ifbibliography
\setlength{\emergencystretch}{3em} % prevent overfull lines
\providecommand{\tightlist}{%
  \setlength{\itemsep}{0pt}\setlength{\parskip}{0pt}}
\setcounter{secnumdepth}{-\maxdimen} % remove section numbering
\usepackage{ragged2e}
\usepackage{color}
\usepackage{listings}
\usepackage{multicol}
\AtBeginSubsection{}

\title{Estadística Inferencial}
\author{Jackson M'coy Romero Plasencia}
\date{Ayacucho 2020}
\institute{\large Universidad Nacional de San Cristóbal de Huamanga \and \normalsize Departamento Académico de Matemática y Física}

\begin{document}
\frame{\titlepage}

\begin{frame}
  \tableofcontents[hideallsubsections]
\end{frame}
\hypertarget{ejercicio-1}{%
\subsection{Ejercicio 1}\label{ejercicio-1}}

\begin{frame}{Ejercicio 1}

\justifying   La distribución de las notas del examen final de Mat. I
resultó ser normal \(N(\mu,\sigma^2)\), con cuartiles 1 y 3 iguales a
6.99 y 11.01 respectivamente.

\begin{itemize}\justifying 
  \item [a.] Determine la media y la varianza de la distribución de las notas.
  \item [b.] Halle el intervalo $[a,b] $ centrado en $\mu$ tal que $P( a\leq\overline{X} \leq b)=0.9544$, donde $\overline{X}$ es la media de la muestra $X_1,X_2,X_3, X_4$ escogida de esa población 
\end{itemize}

\end{frame}

\hypertarget{ejercicio-2}{%
\subsection{Ejercicio 2}\label{ejercicio-2}}

\begin{frame}{Ejercicio 2}

\justifying Un proceso automático llena bolsas de café cuyo peso neto
tiene una media de 250 gramos y un desviación estándar de 3 gramos. Para
controlar el proceso, cada hora se pesan 36 bolsas escogidas al azar; si
el peso neto medio está entre 249 y 251 gramos se continúa con el
proceso aceptando que el peso neto medio es 250 gramos y en caso
contrario, se detiene el proceso para reajustar la máquina.

\begin{itemize}\justifying 
  \item [a.] ¿Cuál es la probabilidad de detener el proceso cuando el peso neto medio realmente es 250 ?
  \item [b.] ¿Cuál es la probabilidad de aceptar que el peso neto promedio es 250 cuando realmente es de 248 gramos ? 
\end{itemize}

\end{frame}

\hypertarget{ejercicio-3}{%
\subsection{Ejercicio 3}\label{ejercicio-3}}

\begin{frame}{Ejercicio 3}

La duración en horas de una marca de tarjeta electrónica se distribuye
exponencialmente con promedio de 1000 horas.

\begin{itemize}\justifying 
  \item [a.] Halle el tamaño $n$ de la muestra de manera que sea 0.9544 la probabilidad de que su media muestral esté entre 800 y 1200 horas.
  \item [b.] Si se obtiene una muestra aleatoria de 100 de esas tarjetas calcular la proabilidad que la duración media de la muestra sea superior a 1100 horas.
\end{itemize}

\end{frame}

\hypertarget{distribuciuxf3n-exponencial}{%
\subsection{Distribución
Exponencial}\label{distribuciuxf3n-exponencial}}

\begin{frame}{Distribución Exponencial}

\[f_X(x,\lambda)=f_X(x)= \frac{\displaystyle e^{\,-x/\lambda} }{\lambda} \quad x\geq0\]
Hallando el \(\mathbb{E}(X)\) y \(\mathbb{V}(X)\). Para trabajar con
integrales, definidas en \(x\geq 0\), vamos a utilizar la functión gamma
\[ \Gamma(\alpha)=\displaystyle\int_{0}^{\infty}\, y^{\alpha-1}\,\displaystyle e^{-y}\,dy =(\alpha-1)!\]

\[\mathbb{E}(X)  = \displaystyle\int_{0}^{\infty}x\, \frac{\displaystyle e^{\,-x/\lambda} }{\lambda}\,dx  = \displaystyle\int_{0}^{\infty} \bigg(\frac{x}{\lambda}\bigg) \, \displaystyle e^{\,-(x /\lambda)} \,dx\]

\[\mathbb{E}(X)  = \displaystyle\int_{0}^{\infty}x\, \frac{\displaystyle e^{\,-x/\lambda} }{\lambda}\,dx =\lambda\displaystyle\int_{0}^{\infty} \bigg(\frac{x}{\lambda}\bigg)^{\color{red}2-1 } \, \displaystyle e^{\,-(x /\lambda)} \,d(x/\lambda)\]

\[\mathbb{E}(X)  =\lambda\Gamma(2)=\lambda(2-1)!=\lambda \]

\end{frame}

\hypertarget{section}{%
\subsection{}\label{section}}

\begin{frame}{}

\[\mathbb{E}(X^2)  = \displaystyle\int_{0}^{\infty}x^2\, \frac{\displaystyle e^{\,-x^/\lambda} }{\lambda}\,dx  = \displaystyle\int_{0}^{\infty} \bigg(\frac{x^2}{\lambda}\bigg) \, \displaystyle e^{\,-(x /\lambda)} \,dx\]

\[\mathbb{E}(X^2)  = \displaystyle\int_{0}^{\infty}x\, \frac{\displaystyle e^{\,-x/\lambda} }{\lambda}\,dx =\lambda^2\displaystyle\int_{0}^{\infty} \bigg(\frac{x}{\lambda}\bigg)^{\color{red}3-1 } \, \displaystyle e^{\,-(x /\lambda)} \,d(x/\lambda)\]

\[\mathbb{E}(X^2)  =\lambda^2\Gamma(3)=\lambda(3-1)!=2\lambda^2 \]
\[\mathbb{V}(X)=\mathbb{E}(X^2)-(\mathbb{E}(X))^2 =2\lambda^2- \lambda^2=\lambda^2\]

\end{frame}

\hypertarget{soluciuxf3n}{%
\subsection{Solución}\label{soluciuxf3n}}

\begin{frame}{Solución}

Ejercicio 1.

\end{frame}

\hypertarget{ejercicio-2-1}{%
\subsection{Ejercicio 2}\label{ejercicio-2-1}}

\begin{frame}{Ejercicio 2}

\begin{enumerate}
[a.]
\item
  \justifying

  \[P(\overline{X}<249|\mu=250)=P\Bigg(\frac{\overline{X}-\mu}{\sigma/\sqrt{n}} <\frac{ 249-250}{3/6}\Bigg) \]
  \[P(\overline{X}<249|\mu=250)=P(Z<-2)=0.02275\]
\end{enumerate}

Probabilidad de detener el proceso es:2(0.02275)= 0.04550026

b.\justifying 
\[P(\overline{X}>249|\mu=248)=P\Bigg(\frac{\overline{X}-\mu}{\sigma/\sqrt{n}} >\frac{ 249-248}{3/6}\Bigg) \]
\[P(\overline{X}>249|\mu=248)=P(Z>2)=P(Z<-2)=0.02275\]

Probabilidad de detener el proceso es:0.02275

\end{frame}

\end{document}
